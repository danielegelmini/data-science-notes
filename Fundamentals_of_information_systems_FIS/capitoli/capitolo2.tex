\chapter{Computational Thinking}
    \textbf{Computational thinking} involves recognizing a problem and transforming it into a form that can be solved by either a computer or a human. The aim is often to automate the process for greater efficiency. A common approach to problem-solving is the \textit{brute-force} method, also known as the naive solution, where every possible option is explored until the correct one is found.
    
    A key concept in computational contexts is the \textbf{protocol}, which refers to a defined set of rules or procedures that govern the way a task or process is carried out, especially in communication or network systems.
    
    Another fundamental aspect is \textbf{generalization}, which is the ability to apply a known method to solve an unknown problem. By leveraging existing strategies or algorithms, we can approach new problems more effectively.
    
    \bigskip
    
    \noindent \textbf{Efficiency} plays a central role in computational thinking because it allows us to save time, resources, and computational power. Efficient algorithms are especially valuable for handling large datasets and completing tasks faster. Evaluating algorithm efficiency often involves analyzing two scenarios:
    
    \begin{itemize}
        \item \textbf{Worst-case scenario:} This occurs when an algorithm takes the maximum amount of time or resources to complete, typically when dealing with the least favorable input. Understanding the worst-case performance is crucial for evaluating an algorithm's limits.
        
        \item \textbf{Average-case scenario:} This refers to the expected time or resources an algorithm takes for a typical input. It gives a more realistic measure of an algorithm's performance in everyday use.
    \end{itemize}
    
    \bigskip
    
    \noindent Computational thinking can be broken down into four main areas:
    
    \begin{itemize}
        \item \textbf{Decomposition:} Breaking down a complex problem into smaller, more manageable sub-problems. This allows each part to be solved independently and then combined to form a complete solution.
        
        \item \textbf{Pattern recognition:} Identifying patterns or trends within a problem or dataset. Recognizing patterns helps predict future outcomes or simplifies the problem by focusing on recurring structures.
        
        \item \textbf{Abstraction:} Reducing the complexity of a problem by focusing on the essential details and ignoring irrelevant information. This makes large-scale problems easier to handle and provides clarity on key aspects.
        
        \item \textbf{Algorithm:} A step-by-step procedure or set of instructions designed to solve a problem or perform a task. Developing an algorithm is a central part of computational thinking, as it provides a structured way to automate solutions.
    \end{itemize}
    
    Together, these four areas provide a systematic approach to solving complex problems, making them suitable for automation by computers.

